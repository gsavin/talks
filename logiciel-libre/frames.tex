\bgimageon{geekpenguin}

\begin{frame}[t]{Une présentation libre\dots}
  Durant cette présentation, vous êtes libre :
  \begin{itemize}
    \item d'intervenir pour y apporter votre contribution ;
    \item de ne pas adhérer à l'esprit du libre ;
    \item mais aussi d'y adhérer !
  \end{itemize}
\end{frame}

\bgimageon{map}

\begin{frame}{Pour aller plus loin\dots}
  \begin{center}
    \tikz
      \node[fill=white, fill opacity=0.75, align=center] {InstallParty organisée par l'association \textsc{GoéLUG}\\au Container (en bas de la CCI), ce soir à 18h30.};
  \end{center}
\end{frame}

\bgimageoff

\section{Librintro}
\subsection{Définition(s)}

\begin{frame}{Libre, d'accord, mais de quoi ?}
  \framesubtitle{Définition de la Free Software Foundation\footnotemark[1]}

  \begin{description}[<+->]
    \item[liberté 0] \small \textbf{d'utiliser} le programme comme vous voulez, pour n'importe quel usage ;
    \item[liberté 1] \small \alert<5>{\textbf{d'étudier} le fonctionnement du programme, et de le modifier pour qu'il effectue vos tâches informatiques comme vous le souhaitez\footnotemark[2] ;}
    \item[liberté 2] \small \textbf{de redistribuer} des copies, donc d'aider votre voisin ;
    \item[liberté 3] \small \alert<5>{\textbf{de distribuer} aux autres des copies de vos versions modifiées ; en faisant cela, vous donnez à toute la communauté une possibilité de profiter de vos changements\footnotemark[2].}
  \end{description}

  \footnotetext[1]{\tiny\url{http://www.gnu.org/philosophy/free-sw.fr.html}}
  \footnotetext[2]{\tiny \alert<5>{l'accès au code source est une condition nécessaire}}
  
  %
  % C'est quoi le code source ?
  %
\end{frame}

\begin{frame}{Libre, d'accord, mais de quoi ?}
  \framesubtitle{"Social Contract" de Debian\footnotemark[1]}

  \begin{itemize}\small
    \item Libre redistribution ;
    \item Code source ;
    \item Travaux dérivés ;
    \item Intégrité du code source de l'auteur ;
    \item Pas de discrimination envers des personnes ou groupes ;
    \item Pas de discrimination envers des domaines ou des entreprises ;
    \item Distribution de la licence ;
    \item La licence ne doit pas être spécifique à Debian ;
    \item La licence ne doit pas contaminer les autres programmes.
  \end{itemize}

  \footnotetext[1]{\tiny\url{https://lists.debian.org/debian-announce/1997/msg00017.html}}
\end{frame}

\begin{frame}{Libre comme une bière, ou comme un discours ?}
  \begin{quote}\centering
    ``Think free as in free speech, not free beer''
  \end{quote}
  \begin{flushright}
    Richard Stallman
  \end{flushright}

  \begin{alertblock}<2->{}\centering
    Libre $\neq$ Gratuit
  \end{alertblock}
\end{frame}

\subsection{Culture Libre}

\begin{frame}{Philosophie}
  \begin{flushright}
    \begin{minipage}{.85\textwidth}
      \begin{quote}\footnotesize
        \og Ce serait peut-être l'une des plus grandes opportunités manquées de notre époque si le logiciel libre ne libérait rien d'autre que du code. \fg{}
      \end{quote}
    \end{minipage}
  \end{flushright}
  
  \begin{itemize}
    \item Culture du partage ;
    \item Rompre le schéma du consommateur ;
    \item Faire ensemble.
  \end{itemize}
  
  % Mouvement du logiciel libre
  % 2004 UNESCO élève le logiciel libre au rang de patrimoine mondial
\end{frame}

\begin{frame}\centering
  En 2004, l'UNESCO élèva le logiciel libre au rang de patrimoine mondial de l'humanité.
\end{frame}

\begin{frame}{Pas que le logiciel !}
  \begin{itemize}
    \item Matériels (open-hardware) ;
    \item Livres ;
    \item Musique ;
    \item Films ;
    \item Données ;
    \item \dots
  \end{itemize}
\end{frame}

\subsection{Pourquoi le Libre ?}

\begin{frame}
  \begin{itemize}
    \item Éthique ;
    \item Interopérabilité (utilisation de formats ouverts) ;
    \item Qualité, fiabilité, sécurité ;
    \item Liberté, égalité, fraternité ;
    \item Pérénnité ;
    \item Adaptabilité.
  \end{itemize}
\end{frame}

\begin{frame}{Interopérabilité}
  \begin{quote}\small
    On entend par compatibilité la capacité de deux systèmes à communiquer sans ambiguïté.
  \end{quote}
  \begin{quote}\small
    On entend par interopérabilité la capacité à rendre compatibles deux systèmes quelconques. L'interopérabilité nécessite que les informations nécessaires à sa mise en œuvre soient disponibles sous la forme de \textbf{standards ouverts}.
  \end{quote}
  \begin{flushright}
    Amendement (rejeté) à la loi DADVSI
  \end{flushright}
\end{frame}

\begin{frame}{Qualité, fiabilité, sécurité}
  \begin{itemize}
    \item Revue par les pairs ;
    \item Processus ouvert ;
  \end{itemize}
  
\end{frame}

\begin{frame}{Adaptabilité}
  Possibilité d'adapter un programme pour des besoins spécifiques (liberté 1).
\end{frame}

\begin{frame}{Dans la vie quotidienne}
  \begin{itemize}
    \item Android (plus ou moins) ;
    \item > 80\%\footnote{\tiny\url{http://w3techs.com/technologies/overview/web_server/all}} des serveurs web sont libres (Apache, Nginx, Lighttpd\dots).
  \end{itemize}
\end{frame}

\begin{frame}{Modèles économiques\footnote{\tiny\url{https://aful.org/professionnels/modeles-economiques-logiciels-libres/differents-modeles}}}
  \begin{itemize}
    \item Services :
    \begin{itemize}
      \item Documentation ;
      \item Prestations ;
      \item Support ;
      \item Certifications ;
    \end{itemize}
    \item Mutualisation ;
    \item Valeurs ajoutées ;
    \item Mécénat.
  \end{itemize}
\end{frame}

\section{Gardiens Protecteurs du Libre}

\subsection{Copyleft}

\begin{frame}{Une notion qui divise : le \textcopyleft{}opyleft}
  Offre les libertés du Libre pour la diffusion d'un logiciel en imposant que les travaux dérivés soient diffusés sous les mêmes conditions.
\end{frame}

\subsection{Licences}

\begin{frame}{La GNU \textbf{G}eneral \textbf{P}ublic \textbf{L}icence}
  \framesubtitle{Historique}
  \tikz[overlay, remember picture]
    \node[anchor=south west] at ($(current page.south west)+(0, 3mm)$) {\pgfuseimage{gpl}};
    
  \begin{description}
    \item[1989] Version 1 écrite par Richard Stallman ;
    \item[1991] Version 2, apparition de la LGPL ;
    \item[2007] Version 3, l'actuelle.
  \end{description}
\end{frame}

\begin{frame}{La GNU \textbf{G}eneral \textbf{P}ublic \textbf{L}icence}
  \framesubtitle{La fratrie}
  
  \begin{description}
    \item[LGPL] \emph{GNU Lesser General Public Licence} ;
    \item[GFDL] \emph{GNU Free Documentation Licence} ;
    \item[AGPL] \emph{GNU Affero General Public Licence}.
  \end{description}
\end{frame}
  
\begin{frame}{La licence BSD\footnote{Berkeley Software Distribution License}}
  \begin{itemize}
    \item Pas de copyleft ;
    \item Proche de la notion de Domaine Public ;
    \item Utilisable dans des logiciels propriétaires ;
    \item Compatibilité à sens unique avec la GPL.
  \end{itemize}
\end{frame}

\begin{frame}{Made in France, la famille CeCILL\footnote{\textbf{CE}A \textbf{C}NRS \textbf{I}NRIA \textbf{L}ogiciel \textbf{L}ibre}}
  \begin{itemize}
    \item Version 1 en 2004 ;
    \item Version 2.1 actuelle en 2013 ;
    \item Garantir le respect du droit français, et du logiciel libre ;
    \item Trois licences :
    \begin{description}
      \item[CeCILL-A] Compatible avec la GPL ;
      \item[CeCILL-B] Compatible avec la BSD ;
      \item[CeCILL-C] Compatible avec la LGPL.
    \end{description}
  \end{itemize}
\end{frame}

\begin{frame}{Et d'autres\dots}
  \begin{itemize}
    \item Licence Apache (non-copyleft) ;
    \item Eclipse Public Licence (copyleft).
  \end{itemize}
\end{frame}

\subsection{Données}

\begin{frame}{Creative Commons}
  \begin{itemize}
    \item Licences sur du contenu.
  \end{itemize}
  
  \begin{block}{Une licence modulable}
    \begin{description}
      \item[BY] Attribution ;
      \item[SA] Partage des conditions initiales à l'identique ;
      \item[ND] Pas de modifications ;
      \item[NC] Pas d'utilisation commerciale
    \end{description}
  \end{block}
  
  \begin{center}
    \begin{tikzpicture}
      \matrix[ampersand replacement=\&] {
        \node {\pgfuseimage{ccze}} ; \&
        \node {\pgfuseimage{ccby}} ; \&
        \node {\pgfuseimage{ccnc}} ; \&
        \node {\pgfuseimage{ccnd}} ; \&
        \node {\pgfuseimage{ccsa}} ; \\
      };
    \end{tikzpicture}
  \end{center}
\end{frame}

\begin{frame}{Creative Commons}
  Sept licences :
  \begin{itemize}
    \item CC-ZERO
    \item CC-BY
    \item CC-BY-SA
    \item CC-BY-ND
    \item CC-BY-NC
    \item CC-BY-NC-SA
    \item CC-BY-NC-ND
  \end{itemize}
\end{frame}

\section{Système d'exploitation libre}

\subsection{La génèse}

\begin{frame}{UNIX}
  \begin{description}
    \item[1969] Ken Thompson développe la première version pour AT\&T\footnote{Laboratoires Bell} ;
    \item[1971] Dennis Ritchie crée le langage C ;
    \item[1977] Bill Joy réalise la première \emph{Berkeley Software Distribution} ;
    \item[1989] Première BSD libre, qui s'affranchit de AT\&T.
  \end{description}
\end{frame}

\begin{frame}{GNU, GNU Not Unix}
  \tikz[overlay, remember picture]
    \node[anchor=south west] at ($(current page.south west)+(0, 3mm)$) {\pgfuseimage{gnu}};
    
  \begin{description}
    \item[1983] Lancement, par Richard Stallman, du projet d'un système d'exploitation compatible UNIX ;
    \item[1985] Création de la Free Software Foundation ;
  \end{description}
\end{frame}

\subsection{Linux}

\begin{frame}{Le noyau Linux}
  \begin{itemize}
    \item Conçu en 1991 par Linus Torvald ;
    \item Inspiré de UNIX ;
    \item Développé en C ;
    \item Libre.
  \end{itemize}
\end{frame}

\begin{frame}{Quelques chiffres}
  \begin{itemize}
    \item 15 803 499 lignes de code (version 3.10) ;
    \item Estimation de 5000 à 6000 développeurs ;
    \item Redévelopper le noyau de manière propriétaire couterait aux alentours de 882M€.
  \end{itemize}
\end{frame}

\subsection{GNU/Linux}

\begin{frame}{Le système d'exploitation GNU/Linux}
  \begin{itemize}
    \item Système d'Exploitation Libre ;
    \item Système GNU prévu pour fonctionner avec \texttt{Hurd} ;
    \item GNU/Linux : Système GNU fonctionnant avec Linux.
  \end{itemize}
\end{frame}

\begin{frame}{De (très) nombreuses distributions}
  \begin{columns}
    \begin{column}[t]{.5\textwidth}
      Les majeures :
      \begin{itemize}
        \item Debian ;
        \item Arch Linux ;
        \item Red Hat ;
        \item SUSE ;
        \item Gentoo ;
        \item Slackware.
      \end{itemize}
    \end{column}
    \begin{column}[t]{.5\textwidth}
      Et bien d'autres :
      \begin{itemize}
        \item Ubuntu ;
        \item Manjaro ;
        \item Linux Mint ;
        \item Fedora ;
        \item \dots
      \end{itemize}
    \end{column}
  \end{columns}
\end{frame}

\begin{frame}{Différents environnements de bureau}
  \begin{multicols}{2}
    \begin{itemize}
      \item GNOME ;
      \item KDE ;
      \item Unity ;
      \item MATE ;
      \item Cinnamon ;
      \item Xfce ;
      \item LXDE ;
      \item Enlightenment.
    \end{itemize}
  \end{multicols}
\end{frame}

\begin{frame}{Différents environnements de bureau}
  \framesubtitle<1>{GNOME}
  \framesubtitle<2>{Enlightenment E17}
  \framesubtitle<3>{KDE}
      
  \begin{tikzpicture}[overlay, remember picture]
    \only<1>{
        \node[anchor=south] at ($(current page.south)+(0,6mm)$) {\pgfuseimage{gnome}};
    }
    \only<2>{
        \node[anchor=south] at ($(current page.south)+(0,6mm)$) {\pgfuseimage{e17}};
    }
    \only<3>{
        \node[anchor=south] at ($(current page.south)+(0,6mm)$) {\pgfuseimage{kde}};
    }
  \end{tikzpicture}
\end{frame}

\begin{frame}{D'autres systèmes d'exploitation libres}
  \begin{itemize}
    \item BSD ;
    \item GNU/BSD (kFreeBSD, NetBSD);
    \item GNU/Hurd ;
    \item Ubuntu Touch ;
    \item Firefox OS.
  \end{itemize}
\end{frame}

\section{Logiciels}

\subsection{Logiciels courants}

\begin{frame}{Quelques exemples\dots}
  \begin{multicols}{2}
    \begin{itemize}
      \item LibreOffice ;
      \item GIMP ;
      \item Inkscape ;
      \item Firefox ;
      \item VLC ;
      \item Apache ;
      \item Emacs ;
      \item Eclipse ;
      \item GPG ;
      \item Blender ;
      \item mplayer ;
      \item GnuCash ;
      \item \dots
    \end{itemize}
  \end{multicols}
\end{frame}

\subsection{Quelques acteurs du libre}

\begin{frame}{Free Software Foundation}
  \tikz[overlay, remember picture]
    \node[anchor=center, opacity=0.25] at ($(current page)+(0, -5mm)$) {\pgfuseimage{fsf}};
  % GNU
  % LibreOffice
  \begin{itemize}
    \item ONG créée en 1985 ;
    \item Promotion du logiciel libre et défense des utilisateurs ;
    \item Financement du projet GNU.
  \end{itemize}
\end{frame}

\begin{frame}{Framasoft}
  \tikz[overlay, remember picture]
    \node[anchor=center, opacity=0.25] at ($(current page)+(0, -6mm)$) {\pgfuseimage{framasoft}};
  
  \begin{itemize}
    \item Réseau français fondé en 2001 ;
    \item Soutenu depuis 2004 par l'association du même nom ;
    \item Promotion, diffusion, et développement de logiciels libres ;
    \item Culture Libre ;
    \item Services libres \texttt{frama*} en ligne : \texttt{-pad}, \texttt{-date}, \texttt{-calc}, \texttt{-sphère}, et bien d'autres.
  \end{itemize}
\end{frame}

\begin{frame}{Mozilla}
  \tikz[overlay, remember picture]
    \node[anchor=center, opacity=0.25] at ($(current page)+(0, -5mm)$) {\pgfuseimage{mozilla}};
  
  \begin{itemize}
    \item Projet fondé en 1998 par Netscape ;
    \item Soutenu depuis par la Fondation Mozilla ;
    \item Navigateur Firefox  ;
    \item Messagerie Thunderbird ;
    \item Mozilla Developer Network ;
    \item Firefox OS.
  \end{itemize}
\end{frame}

\begin{frame}{Et bien d'autres\dots}
  \begin{itemize}
    \item April ;
    \item Open Source Initiative ;
    \item Fondation Linux ;
    \item Canonical (Ubuntu);
    \item The Document Foundation (LibreOffice) ;
    \item \dots
  \end{itemize}
\end{frame}

\let\origaddtocontents=\addtocontents
\def\dontaddtocontents#1#2{}

\let\addtocontents=\dontaddtocontents
\section*{Invisible Section}
\let\addtocontents=\origaddtocontents

\bgimageon{tux}

\begin{frame}{Pour aller plus loin\dots}
  \centering
  \begin{beamercolorbox}[sep=8pt,center,shadow=false,rounded=true]{title}
    \usebeamerfont{title}Discussion / Manipulation\par%
  \end{beamercolorbox}
  \vfill
  \tikz \node[fill=white, fill opacity=0.75, inner sep=2mm, rounded corners=3pt] {Merci de votre attention.};
\end{frame}

\bgimageoff

\begin{frame}
  Sources :
  \begin{itemize}\small
    \item \emph{La philosophie du logiciel libre}, par Jean-François \textsc{Becquet} ;
    \item \emph{Qu'est ce que le logiciel libre et pourquoi faire ?}, par Dave Null ;
    \item \textsc{Wikipedia} ;
    \item \emph{Free Culture}, Lawrence Lessig ;
  \end{itemize}
  
  Sources :
  \begin{center}\footnotesize
    \href{https://github.com/gsavin/talks/tree/master/logiciel-libre}{\texttt{https://github.com/gsavin/talks/}}
  \end{center}
  
  \tikz[overlay, remember picture]\node[anchor=south west] at ($(current page.south west)+(1mm, -0.6mm)$) {\pgfuseimage{licence}};
\end{frame}
